\chapter*{}
\vspace*{-1.0cm}
\begin{center}
\addcontentsline{toc}{chapter}{\protect\numberline{}ABSTRAK}
\normalfont\LARGE\textbf{ABSTRAK}
\end{center}
% Penelitian ini menyajikan kerangka kerja forensik proaktif berbasis manajemen log untuk meningkatkan kesiapan forensik dan memastikan keandalan bukti log dalam lingkungan Ujian Kemampuan Bahasa Inggris (English Proficiency Test/EPT) online. Kerangka kerja ini mencakup pengumpulan log otomatis, penyimpanan yang terorganisir, dan analisis mendekati waktu nyata untuk mengatasi tantangan terkait pelestarian bukti dan deteksi anomali. Dengan mengikuti pedoman NIST 800-92, sistem ini memastikan penanganan log yang aman dan pengelolaan log secara tersentralisasi, sehingga mengurangi risiko kehilangan data. Melalui integrasi fitur pemantauan proaktif dan pelaporan, seperti notifikasi otomatis dan dashboard deteksi anomali, sistem ini memfasilitasi identifikasi dan respons cepat terhadap aktivitas mencurigakan. Hasil penelitian menyoroti efektivitas kerangka kerja ini dalam menyederhanakan proses investigasi forensik, meningkatkan kegunaan log, dan menjaga integritas ujian online. Pendekatan ini berkontribusi pada kemajuan kesiapan forensik di institusi pendidikan, menawarkan solusi yang skalabel dan efisien untuk menjaga kredibilitas bukti digital.\\\\
\textbf{Kata kunci:} 