\chapter{PRESENTATION, ANALYSIS AND INTERPRETATION OF DATA}
In thesis writing, the most difficult part to defend is chapter 4 because it is in this section where you will present the results of the whole study. Here is a sample thesis format.
\begin{figure}[H] 
	\centering
	\includegraphics[width=14cm]{figure/structure menu.png}
	\caption{Log Management Structure Menu}
	\label{fig:structure-menu}
\end{figure}

The picture above \ref{fig:structure-menu} is a menu structure. the menu structure is used to display the structure of the log management application.
% =========================================================
\section{Phase 1}

\subsection{Define Problem Statement}
% \begin{figure}[H] 
	%     \centering
	%     \includegraphics[width=12cm]{figure/attack-defence-tree-model-online-exam.png}
	%     \caption{Attack Defend Tree Model \citet{rosmansyah2019attackdefensetreeonaeexamsystem}}
	%     \label{fig:attack-defend-tree-model}
	% \end{figure}

\begin{itemize}
	\item Server Attack
	\item Cheating Attack
	\item Software Attack (e.g., errors in the application)
	\item Client Device Attack
	\begin{itemize}
		\item Network Device Attack
		\item Client Device Attack
	\end{itemize}
	\item Officer Attack (e.g., attack on proctor)
	\item Examinee Attack
\end{itemize}
List above \citet{rosmansyah2019attackdefensetreeonaeexamsystem} is attack-defence tree model systematically illustrates how various forms of attacks can occur in an online exam system and how defence measures can be integrated at each potential attack point. With this model, exam organizers can be better prepared to anticipate threats and ensure the continuity and security of the online examination process.
\begin{figure}[H] 
	\centering
	\includegraphics[height=18cm]{figure/df-evidence-loss.png}
	\caption{Digital Forensic Evidence Loss \cite{larchenko2025mindmap}}
	\label{fig:df-evidence-loss}
\end{figure}

One of the major risks in digital forensic investigations is the loss of critical evidence due to various technical and procedural factors. These losses can compromise the integrity, completeness, and legal admissibility of evidence, thereby weakening the overall investigation. Evidence loss may occur during collection, transmission, storage, or analysis phases—often unnoticed until a post-incident review. Therefore, proactive mechanisms must be in place to preserve, validate, and monitor digital logs systematically.

Phase 1 of the proposed framework focuses on defining the core objectives and technical requirements, as introduced in Chapter 3. This phase emphasizes the proactive nature of digital forensic readiness, where logs are collected, preserved, and structured long before any examination session or security incident occurs.



This phase ensures that the framework's direction remains aligned with long-term forensic preparedness, enabling timely evidence collection and minimizing the risk of missing critical data when incidents do occur.
\subsection{Research Objectives}
The main objective is to build a centralized and automated log management system that enables continuous monitoring and log acquisition in an online learning environment. The system should not wait for anomalies or incidents to occur, but instead establish an audit trail that is always active and available for future analysis.

Key system requirements identified in this phase include:

\begin{itemize}
	\item Continuous log acquisition from key components such as Moodle, the operating system, database services, and web servers.
	\item A proactive scheduling mechanism (e.g., \texttt{crontab}) to retrieve logs at defined intervals, regardless of exam schedules.
	\item Centralized log storage with access control, timestamping, and integrity verification using cryptographic hashes.
	\item Compatibility with scalable infrastructure (e.g., Azure VMSS) to support dynamic resource allocation.
	\item Capability to export logs in a structured format (CSV/JSON) for downstream forensic processes.
	\item Readiness for future integration with machine learning-based analysis.
\end{itemize}


\subsection{Research Method}
To achieve the objectives outlined in the previous section, a structured method was designed based on a system development approach, incorporating best practices from digital forensic readiness and proactive log management frameworks. The methodology integrates technical implementation, experimental testing, and expert validation to ensure both functional performance and forensic reliability.

The core steps of the method are as follows:

\begin{itemize}
	\item \textbf{Framework Design:} A modular architecture is designed, comprising components such as a log collector, scheduler, centralized log storage, machine learning engine, dashboard interface, and notification system. The design adheres to log lifecycle principles based on NIST SP 800-92.
	\item \textbf{Data Flow Implementation:} A data pipeline is developed to enable automated log acquisition, transmission using \texttt{rsync}, and storage into structured directories. The logs include user quiz attempts, question interaction steps, and system events from the Moodle platform.
	\item \textbf{Machine Learning Integration:} An unsupervised anomaly detection model (Isolation Forest) is trained using historical exam data to classify suspicious activity based on exam duration and scoring patterns.
	\item \textbf{Preservation and Reporting:} Log integrity is ensured through MD5 hashing and timestamping. Reports are generated in PDF format to assist investigators and administrators in analyzing flagged incidents.
	\item \textbf{Validation and Testing:} The system is tested in a simulated environment using real log data from an online English Proficiency Test (EPT), and the results are validated through expert review by digital forensic practitioners.
\end{itemize}

This method ensures that the framework is not only technically viable but also compliant with forensic principles of evidence preservation, integrity, and traceability. The combination of automation, machine learning, and structured log management offers a holistic solution for supporting proactive digital forensics in online examination systems.

\section{Phase 2}
Phase 2 focuses on conducting a comprehensive literature review to identify best practices, frameworks, and standards relevant to digital forensics in online education environments. This stage is critical for ensuring that the proposed framework is grounded in established knowledge and is capable of addressing current gaps in proactive forensic readiness.

The literature review emphasizes two main areas: (1) digital forensic frameworks, and (2) log management standards. Particular attention is given to the NIST Special Publication 800-92, which provides detailed guidelines for computer security log management. This standard outlines principles for log generation, collection, transmission, storage, analysis, and disposal, and serves as the core reference model for the framework in this study.

In addition, previous research by Adel et al \cite{adel2024ethicore}. and Alharbi et al\cite{proactiveandreactivedigitalforensics}. was reviewed to understand approaches in proactive and reactive forensics, particularly in e-learning and cloud-based environments. These studies highlighted the importance of early evidence acquisition, modular framework design, and integration with real-world systems such as LMS and cloud platforms.

Key insights extracted from this phase include:

\begin{itemize}
	\item The necessity of proactive log acquisition to minimize evidence loss.
	\item Integration of log lifecycle processes with operational systems.
	\item Modular design that allows future expansion into anomaly detection and incident response.
	\item Challenges in standardizing log formats across heterogeneous sources.
\end{itemize}

This literature review provided both theoretical and practical direction for designing a customized framework suitable for scalable, secure, and evidence-ready examination systems. It also helped define the scope and limitations to be addressed in the design phase that follows.
\begin{table}[H]
	\centering
	\caption{Summary of Related Works and Relevance to This Research}
	\label{tab:literature_review}
	\begin{tabular}{|p{3.5cm}|p{4.5cm}|p{4.5cm}|}
		\hline
		\textbf{Author(s)} & \textbf{Focus of Study} & \textbf{Relevance to Proposed Framework} \\
		\hline
		Kent et al. (NIST SP 800-92, 2006) \cite{kentnist800922006guide} & Guidelines for log generation, collection, transmission, storage, and disposal & Serves as the primary reference for structuring the proactive log management phases \\
		\hline
		Adel et al. (2021) \cite{adel2024ethicore} & Modular digital forensic framework for e-learning environments & Provides foundational model for framework phases, adapted and expanded in this study \\
		\hline
		Alharbi et al. (2020) \cite{proactiveandreactivedigitalforensics} & Proactive and reactive digital forensic models in cloud-based systems & Supports the concept of readiness and proactive logging in scalable infrastructure \\
		\hline
		Smirani and Boulahia (2022) \cite{smirani2022algorithm} & Evaluation metrics for machine learning in intrusion/anomaly detection & Guides the measurement of ML performance (accuracy, precision, recall, F1) in Phase 6 \\
		\hline
		Garg and Goel (2023) \citet{garg2023preserving} & Application of ML for log evidence in academic integrity cases & Validates the use of Isolation Forest for detecting abnormal behavior in online exams \\
		\hline
	\end{tabular}
\end{table}

\section{Phase 3}
Phase 3 focuses on designing and customizing the forensic log management framework according to the objectives and insights gathered in the previous phases. This design stage emphasizes a modular, scalable, and proactive architecture aligned with the principles of NIST SP 800-92 and adapted from the framework structure of Adel et al.

The design process begins by mapping the nine essential log management processes—ranging from log identification to reporting—into functional components within the online examination ecosystem. Each component is defined to operate independently yet cohesively to support end-to-end forensic readiness.

Key modifications and design decisions include:

\begin{itemize}
	\item \textbf{Modular architecture}: The framework is divided into clear modules such as log acquisition, transmission, storage, analysis, notification, and preservation. This allows independent testing and scaling.
	
	\item \textbf{Cloud deployment readiness}: The infrastructure is built using Azure Virtual Machine Scale Sets (VMSS) to support dynamic provisioning and load balancing during examination periods.
	
	\item \textbf{Integration with Moodle LMS}: The system is directly linked to Moodle’s internal logging and quiz attempt tracking mechanisms to enable real-time collection of learning and assessment activities.
	
	\item \textbf{Security and traceability}: The framework includes features for log timestamping, structured storage, and integrity verification using MD5 checksums to maintain evidential reliability.
	
	\item \textbf{Preparation for downstream analysis}: Logs are formatted in a machine-readable structure (CSV/JSON) to support future use in anomaly detection and forensic reporting.
\end{itemize}

Figure~\ref{fig:framework-proposed} illustrates the overall framework design adapted from NIST SP 800-92, enhanced to suit the specific requirements of online exam environments. This architecture ensures that the framework remains extensible and can be evaluated under simulated forensic scenarios in later phases.

\section{Phase 4}
Phase 4 involves the implementation and simulation of the designed forensic log management framework within a controlled test environment. This phase validates whether the architectural design and functional components can operate effectively in practice, especially in the context of online examination scenarios.

The simulation was conducted using a testbed environment that replicates the infrastructure of an actual university's online exam system.

\subsection{Log Identification}
\begin{figure}[H] 
	\centering
	\includegraphics[height=17cm]{figure/log-source.png}
	\caption{Log Source}
	\label{fig:log-source}
\end{figure}
The diagram \ref{fig:log-source} represents the result of an identification process of various log sources within the system environment. These sources are categorized into two main levels: \textbf{infrastructure level} and \textbf{application level}.


\begin{longtable}{|p{3.5cm}|p{5cm}|p{7cm}|}
	\caption{Table Log Source, Description and Example Log Data}
	\label{tab:log_sources} \\
	\hline
	\textbf{Log Source} & \textbf{Description} & \textbf{Example Log Data} \\
	\hline
	\endfirsthead
	
	\hline
	\textbf{Log Source} & \textbf{Description} & \textbf{Example Log Data} \\
	\hline
	\endhead
	
	Trigger Event Security & Log entries triggered by Cloudflare security rules (e.g. WAF, bot protection) &
	\scriptsize\ttfamily
	\begin{lstlisting}
		EventID: 38492 
		Type: SQL Injection 
		Severity: High 
		URI: /login/index.php
	\end{lstlisting}
	\normalsize \\
	\hline
	
	Log Explore Query & Cloudflare query logs from Log Explorer for audit and troubleshooting &
	\scriptsize\ttfamily
	\begin{lstlisting}
		query="SELECT * 
		FROM traffic_logs 
		WHERE status=403"
	\end{lstlisting}
	\normalsize \\
	\hline
	
	Apache Access Log & Log of all HTTP requests handled by Apache web server &
	\scriptsize\ttfamily
	\begin{lstlisting}
		192.0.2.1 - - [21/Jun/2025:10:04:32 +0000] 
		"GET /moodle/login/index.php HTTP/1.1" 200 4523
	\end{lstlisting}
	\normalsize \\
	\hline
	
	Apache Error Log & Application/server-side error messages generated by Apache &
	\scriptsize\ttfamily
	\begin{lstlisting}
		[Sat Jun 21 10:05:12.123456 2025] 
		[php:error] [client 192.0.2.1] 
		PHP Fatal error: Call to undefined function...
	\end{lstlisting}
	\normalsize \\
	\hline
	
	System Logs (Syslog) & General OS-level events such as service starts, reboots, errors &
	\scriptsize\ttfamily
	\begin{lstlisting}
		Feb 26 06:25:02 localhost rsyslogd: 
		[origin software="rsyslogd" swVersion="8.32.0" 
		x-pid="1332" x-info="http://www.rsyslog.com"] 
		rsyslogd was HUPed
	\end{lstlisting}
	\normalsize \\
	\hline
	
	MySQL Logs & Database-level warnings and errors from MySQL error log &
	\scriptsize\ttfamily
	\begin{lstlisting}
		2023-02-24T16:27:52.714361Z 0 [Warning] 
		Could not increase number of max_open_files 
		to more than 5000 (request: 50000)
	\end{lstlisting}
	\normalsize \\
	\hline
	
	PHP-FPM Logs & Runtime messages related to PHP process management (PHP 7.2) &
	\scriptsize\ttfamily
	\begin{lstlisting}
		[08-Nov-2023 12:52:10] NOTICE: 
		systemd monitor interval set to 10000ms
	\end{lstlisting}
	\normalsize \\
	\hline
	
	Log Quiz Attempts & Records each attempt made by a user on a quiz including layout, timing, and score &
	\scriptsize\ttfamily
	\begin{lstlisting}
		attempt_id: 1028782
		user_id: 34413
		name: user
		course: EPT Home Edition
		quiz: Grammar
		uniqueid: 1033212
		layout: 1,2,3,...,41,0
		timestart: 1711939947
		timefinish: 1711941217
		score: 11
	\end{lstlisting}
	\normalsize \\
	\hline
	
	Log Quiz Steps & Step-by-step interaction data per quiz question attempt, including timing and score &
	\scriptsize\ttfamily
	\begin{lstlisting}
		attempt_id: 1028782
		user_id: 34413
		quiz: Grammar
		question_attempt_id: 19613856
		step_id: 64023090
		step_state: todo
		step_start_time: 1711939947
		next_step_time: 1711940179
		time_spent_on_question: 232
	\end{lstlisting}
	\normalsize \\
	\hline
	
	Log Course Activity & Activity logs related to course views, resources accessed, etc. &
	\scriptsize\ttfamily
	\begin{lstlisting}
		Time: 23/06/25, 10:49
		User full name: admin admin
		Event context: Course: EPrT HE Pre-Exam
		Event name: Course viewed
		IP address: 103.233.100.202
	\end{lstlisting}
	\normalsize \\
	\hline
	
	User Logs & Tracks user actions across the platform (login, logout, enroll, etc.) &
	\scriptsize\ttfamily
	\begin{lstlisting}
		Time: 2025-06-21 09:55:00 
		User: student02 
		Action: loggedout
	\end{lstlisting}
	\normalsize \\
	\hline
	
\end{longtable}


Log identification is done by identifying the sources that generate logs by reviewing the results from exam participants. The log attempt results will be stored in a database, therefore the log source is located in the MySQL database.
In the process of log analysis related to online examinations in Moodle, several database tables have been identified as essential sources of information for understanding user behavior and quiz performance. These tables are involved in recording quiz attempts and linking them to user profiles, course structures, and module instances.

The key identified tables include:

\begin{itemize}
	\item \textbf{mdl\_quiz\_attempts}: This table stores individual quiz attempt records by users. It contains timestamps for when an attempt started and finished, along with the score achieved and the state of the attempt (e.g., in progress, finished).
	
	\item \textbf{mdl\_user}: This table contains user-specific information such as user ID, first name, last name, and authentication details. It allows quiz attempts to be linked directly to the participants.
	
	\item \textbf{mdl\_quiz}: This table stores metadata about each quiz, including its name, settings, grading method, and association to a course.
	
	\item \textbf{mdl\_course}: This table defines course-level information. It helps in organizing and grouping quizzes under the appropriate academic or training program context.
	
	\item \textbf{mdl\_course\_modules}: This table links quizzes to specific module instances within a course, enabling fine-grained selection of quiz activity based on module ID. It is critical for identifying which quizzes are deployed in which course sections.
\end{itemize}

Together, these tables provide a comprehensive relational schema for extracting, preprocessing, and analyzing quiz attempt data in a structured and meaningful way.
\begin{figure}[H] 
	\centering
	\includegraphics[width=14cm]{figure/erd-database-quiz-attempts.png}
	\caption{ERD from Several Column Tables}
	\label{fig:erd-tables}
\end{figure}
The Entity Relationship Diagram (ERD) presented above illustrates the core database schema used to represent online exam results within the Moodle learning management system.

\begin{figure}[H] 
	\centering
	\includegraphics[width=14cm]{figure/sandbox-log-eprt.png}
	\caption{Backup from mysql dump}
	\label{fig:sandbox-mysql-dump}
\end{figure}

Fig \ref{fig:sandbox-mysql-dump} is the quiz attempts of exam participants. This data will be utilized for analyzing the behavior of the exam participants.

% \begin{figure}[H] 
	%     \centering
	%     \includegraphics[width=14cm]{figure/example-log-quiz-attempts.png}
	%     \caption{Log data from mysql quiz attemps}
	%     \label{fig:logs-mysql=quiz=attempts}
	% \end{figure}

\begin{longtable}{|p{1.3cm}|p{1.5cm}|p{1.2cm}|p{2.1cm}|p{1.5cm}|p{1.5cm}|p{1.8cm}|p{1.8cm}|p{0.8cm}|}
	\caption{Course Log Records from User Attempts Quiz} \label{tab:quiz-attempts-log} \\
	\hline
	\textbf{Attempt ID} & \textbf{User ID} & \textbf{Name} & \textbf{Course} & \textbf{Quiz} & \textbf{Layout} & \textbf{Start Time} & \textbf{Finish Time} & \textbf{Score} \\
	\hline
	\endfirsthead
	
	\hline
	\textbf{Attempt ID} & \textbf{User ID} & \textbf{Name} & \textbf{Course} & \textbf{Quiz} & \textbf{Layout} & \textbf{Start Time} & \textbf{Finish Time} & \textbf{Score} \\
	\hline
	\endhead
	
	1028782 & 34413 & Anon & EPrT Home Edition & Grammar & 41 items & 1711939947 & 1711941217 & 11 \\
	\hline
	1028779 & 33532 & Anon & EPrT Home Edition & Grammar & 41 items & 1711939560 & 1711940973 & 18 \\
	\hline
	1028776 & 33601 & Anon & EPrT Home Edition & Grammar & 41 items & 1711939379 & 1711940864 & 13 \\
	\hline
	1028773 & 33718 & Anon & EPrT Home Edition & Grammar & 41 items & 1711939355 & 1711940774 & 29 \\
	\hline
	
\end{longtable}

\begin{landscape}
	\begin{longtable}{|p{1.4cm}|p{2cm}|p{1.3cm}|p{2.5cm}|p{1.8cm}|p{3cm}|p{4.2cm}|p{1.2cm}|p{2.5cm}|}
		\caption{Course Log Records from Moodle Course} \label{tab:course-log} \\
		\hline
		\textbf{Time} & \textbf{User full name} & \textbf{Affected user} & \textbf{Event context} & \textbf{Component} & \textbf{Event name} & \textbf{Description} & \textbf{Origin} & \textbf{IP address} \\
		\hline
		\endfirsthead
		
		\hline
		\textbf{Time} & \textbf{User full name} & \textbf{Affected user} & \textbf{Event context} & \textbf{Component} & \textbf{Event name} & \textbf{Description} & \textbf{Origin} & \textbf{IP address} \\
		\hline
		\endhead
		
		23/06/25, 10:51 & admin admin & - & Course: EPrT HE Pre-Exam & Logs & Log report viewed & The user with id '4' viewed the log report for the course with id '4'. & web & 103.233.100.202 \\
		\hline
		23/06/25, 10:51 & admin admin & - & Course: EPrT HE Pre-Exam & Logs & Log report viewed & The user with id '4' viewed the log report for the course with id '4'. & web & 103.233.100.202 \\
		\hline
		23/06/25, 10:51 & admin admin & - & Course: EPrT HE Pre-Exam & Logs & Log report viewed & The user with id '4' viewed the log report for the course with id '4'. & web & 103.233.100.202 \\
		\hline
		23/06/25, 10:49 & admin admin & - & Course: EPrT HE Pre-Exam & System & Course viewed & The user with id '4' viewed the course with id '4'. & web & 103.233.100.202 \\
		\hline
		
	\end{longtable}
\end{landscape}

Fig \ref{tab:quiz_attempt_log} is logs data quiz attempts. By analyzing the collected data, valuable insights can be gained into the behavior of users during exams. One such insight can be gleaned from the time participants take to complete the exam.
\subsection{Proactive Collection}
In modern system administration, the use of \textbf{cron job} on Linux/Unix systems is a prevalent method for performing automated, periodic tasks such as log extraction, data transformation, or backup. Cron enables precise scheduling through the \texttt{crontab} utility, which defines job frequency using five time fields: minute, hour, day, month, and weekday \citet{davidovic2015cron}.

Academic validation of this method is found in related research. For instance, Hazwam et al. propose a cron-triggered Perl script to periodically parse honeypot logs into a database, significantly reducing storage requirements and improving system performance \citet{halim2019honeypot}. This demonstrates the effectiveness of cron-based scheduling for resource-efficient log handling.

Furthermore, system reliability and auditing are supported by best practices from NIST SP 800‑92, which emphasize the scheduling of log management routines for integrity checks and retention.

In summary, \textbf{cron-based periodic log collection}  offers a lightweight, scalable, and verifiable method for preserving and processing database-derived logs, and is validated both in the field and in academic literature.

Proactive collection for automated collection. automated collection systematic for collecting and storing data or evidence before incidents occur that could lead to data loss. There are three types of logs that will be collected: attempt logs, quiz attempt step logs, and course activity logs.
\begin{verbatim}
	0 18 * * * * /home/mirage/tes-dump-sql/export-sql.sh
\end{verbatim}
\begin{figure}[H] 
	\centering
	\includegraphics[width=14cm]{figure/scheduler-crontab.png}
	\caption{Scheduler for proactive collection}
	\label{fig:scheduler}
\end{figure}


Figure~\ref{fig:scheduler} shows the job scheduler used for periodic log collection. It automates the retrieval of log data from the database and system components, ensuring timely collection and synchronization to centralized storage to support proactive forensic.To extract detailed quiz attempt logs for further analysis, a structured SQL query was designed to retrieve records from multiple interconnected tables within the Moodle database schema. The focus of this query is to obtain finished quiz attempts from specific course modules associated with the quiz activity.

To facilitate log analysis related to online examinations within the Moodle platform, a specific SQL query was constructed to extract quiz attempt data based on the associated course identifier (course id). This approach ensures that only relevant records tied to a particular course and quiz activity are processed, improving both performance and precision in downstream analytics.
\subsection{Log Transmission}
To support secure and efficient log transmission, the system utilizes the \texttt{rsync} protocol to transfer collected log files from the master virtual machine (VM), which hosts the examination platform, to a centralized log storage server. The use of \texttt{rsync} enables incremental synchronization, ensuring that only updated or newly generated log data is transmitted, thereby reducing bandwidth usage and improving transfer speed. This transmission occurs as part of a scheduled routine executed daily after examination sessions conclude, ensuring timely backup and availability of log data for further analysis and archival.

Rsync was selected over SCP for log transmission because it supports delta transfer, checksum verification, and transfer resumption—features that ensure both performance and integrity during multiple, incremental log backups. In contrast, SCP lacks these forensic-grade guarantees~\cite{nussbaum2012performance,superuser_rsync_vs_scp,stackexchange_rsync_scp_checksum}.


\begin{figure}[H] 
	\centering
	\includegraphics[width=14cm]{figure/export-sql.png}
	\caption{Exporter script}
	\label{fig:exporter}
\end{figure}

A scheduled task is configured using the Linux \texttt{crontab} utility to automate daily log collection at 18:00 local time. This ensures logs are consistently backed up at the end of each examination session. exporter software component used to collect and transfer data from one system or virtual machine (VM) to another. By using exporters, organizations can automate data workflows, centralize information, and gain valuable insights from their systems. Whether for monitoring, logging, or data migration.

\begin{figure}[H] 
	\centering
	\includegraphics[width=14cm]{figure/ml-workflow.png}
	\caption{Machine learning workflow}
	\label{fig:mlops}
\end{figure}

Machine learning workflow \ref{fig:mlops} which will read from the vm-b database. Data from the vm-b database will be processed by the machine learning model that has been created.

\subsection{Log Storage}
In the context of proactive forensic log management, the log storage component plays a critical role in ensuring data availability, integrity, and traceability over time. Logs that have been proactively collected and transmitted must be preserved in a structured and secure manner to support future investigations, audits, and potential legal proceedings.

\begin{figure}[H] 
	\centering
	\includegraphics[width=14cm]{figure/log-backup-sql.png}
	\caption{Log data backup}
	\label{fig:logs-backup-linux}
\end{figure}

Figure \ref{fig:logs-backup-linux} To ensure efficient log management and prevent data loss, user activity data from the MySQL database related to the English Proficiency Test (EPT) is collected daily and stored in a centralized system. Each log entry is timestamped for accurate tracking, while strategies like log rotation, compression, and regular backups safeguard against data loss. This streamlined approach optimizes storage, enhances accessibility, and ensures the availability of critical data for forensic analysis.

\subsection{Analysis of the Data}

The data collection process has been completed. The next step involves analyzing the gathered data, specifically the data of exam participants who have attempted the EPRT quiz. Two key indicators will be considered:


\begin{table}[H]
	\centering
	\renewcommand{\arraystretch}{1.3}
	\begin{tabular}{|p{4cm}|p{10cm}|}
		\hline
		\textbf{Column Name} & \textbf{Example of Data} \\
		\hline
		attempt\_id & 1029126 \\
		\hline
		id & 76022 \\
		\hline
		firstname & NURUL \\
		\hline
		lastname & IZZAH LUTHFIAH NUR \\
		\hline
		course\_name & EPrT Home Edition \\
		\hline
		quiz\_name & Grammar \\
		\hline
		uniqueid & 1033556 \\
		\hline
		layout & 1,0,2,3,4,5,6,0,7,8,9,10,11,0,12,13,14,15,16,0... \\
		\hline
		timestart & 2024-04-02 07:39:53 \\
		\hline
		timefinish & 2024-04-02 08:04:29 \\
		\hline
		score & 26 \\
		\hline
		diff\_time & 0 days 00:24:36 \\
		\hline
		diff\_time\_minute & 24.600000 \\
		\hline
		epoch\_start & 1712043593 \\
		\hline
		epoch\_finish & 1712045069 \\
		\hline
		time\_diff\_seconds & 1476 \\
		\hline
		anomaly & 1 \\
		\hline
	\end{tabular}
	\caption{Example of column names and corresponding data in the online exam anomaly detection system}
	\label{tab:anomaly_data}
\end{table}

Table~\ref{tab:anomaly_data} presents a sample of the data used in the anomaly detection system for online examinations utilizing the \textit{Isolation Forest} algorithm. The dataset consists of various attributes that represent the participants' activity during the exam, ranging from identity information to time-based metrics and performance results.

Key attributes used include:

\begin{itemize}
	\item \textbf{attempt\_id}, \textbf{id}, and \textbf{uniqueid}: Unique identifiers for each exam session.
	\item \textbf{firstname} and \textbf{lastname}: The name of the exam participant.
	\item \textbf{course\_name} and \textbf{quiz\_name}: Indicate the course title and the type of quiz taken.
	\item \textbf{layout}: Stores the sequence of questions accessed by the participant during the exam.
	\item \textbf{timestart} and \textbf{timefinish}: Represent the start and end timestamps of the exam attempt.
	\item \textbf{diff\_time}, \textbf{diff\_time\_minute}, and \textbf{time\_diff\_seconds}: Metrics that capture the total time taken to complete the exam in various units.
	\item \textbf{score}: The final score obtained by the participant on the quiz.
	\item \textbf{anomaly}: The result of anomaly detection, where a value of \texttt{1} indicates that the data is classified as anomalous, and \texttt{0} means it is considered normal.
\end{itemize}


\begin{table}[H]
	\centering
	\caption{Features Used for Analysis Data}
	\begin{tabular}{|l|p{10cm}|c|}
		\hline
		\textbf{Feature} & \textbf{Details} & \textbf{Data Value} \\
		\hline
		\textbf{Time\_taken} & Time taken by examinee to finish the session by calculating the difference of timestamp between examinee’s start time and finish time & 00:19:14 \\
		\hline
		\textbf{score} & Score of an examinee in a session, this represents how many questions the examinee answered correctly & 22 \\
		\hline
	\end{tabular}
\end{table}


\begin{figure}[H] 
	\centering
	\includegraphics[height=14cm]{figure/flow-ml-model.png}
	\caption{Flowchart ML Training }
	\label{fig:flow-ml-training}
\end{figure}
Figure~\ref{fig:flow-ml-training} illustrates the overall process involved in training a machine learning (ML) model for proactive forensic analysis. The flowchart begins with the acquisition of training data, typically in the form of log files or structured events collected during simulated or real-world scenarios. These datasets undergo preprocessing steps, which may include cleaning, normalization, and feature extraction to ensure they are suitable for model input.

The machine learning model was trained using a dataset comprising 223 labeled instances. To ensure reliable evaluation and prevent class imbalance, the dataset was partitioned using an 80:20 split ratio with stratification. Specifically, 178 instances were allocated for training (\texttt{X\_train}, \texttt{y\_train}), and 45 instances for testing (\texttt{X\_test}, \texttt{y\_test}). The use of \texttt{stratify=y} ensured that the class distribution in both training and testing subsets remained proportional to the original dataset.



\begin{figure}[H] 
	\centering
	\includegraphics[height=14cm]{figure/flow-analysis-data.png}
	\caption{Flowchart Log Analysis }
	\label{fig:flow-analysis-log}
\end{figure}

In previous research using machine learning to analyze the data log evidence in online exam \citet{garg2023preserving}. Furthermore, to tackle the ongoing challenge of establishing the ground truth in cases of academic dishonesty.

An anomaly detection model was developed using the Isolation Forest algorithm, which is well-suited for identifying outliers in high-dimensional datasets. The model was configured with a contamination rate of \texttt{0.005}, indicating that approximately 0.5\% of the data is assumed to be anomalous. This low contamination value reflects the expectation that only a very small portion of the dataset represents abnormal behavior.

To enhance the robustness of anomaly detection, the model utilizes \texttt{200} estimators (\texttt{n\_estimators}), meaning that 200 isolation trees were built during the training process. The \texttt{max\_samples} parameter was set to \texttt{0.8}, allowing each tree to be trained on 80\% of the available data, which introduces diversity among trees and improves generalization. Additionally, the model uses \texttt{max\_features} set to \texttt{0.75}, meaning that only 75\% of the total features are considered when constructing each tree, further increasing randomness and reducing overfitting.

Finally, the random state was fixed at \texttt{42} to ensure reproducibility across multiple training sessions. This configuration balances sensitivity to rare anomalies with model stability, making it suitable for detecting suspicious behavior in forensic log data.


Use of unsupervised algorithms because the dataset used is unlabeled. The dataset is taken from campuses located in Indonesia with online exams. The dataset used for training machine learning from the exam results on April 1 to 5, 2023. Number of datasets used 223.

\begin{equation}
	\text{Accuracy} = \frac{TP + TN}{TP + TN + FP + FN}
\end{equation}
Accuracy is the percentage of correct predictions that a learner has achieved. It is computed by dividing the number of correct estimates by the total number of prediction \citet{smirani2022algorithm}.
\begin{equation}
	\text{Precision} = \frac{TP}{TP + FP}
\end{equation}
Precision also known as the positive predictive value, is the ratio of the pertinent instances to the retrieved instances \citet{smirani2022algorithm}.

\begin{equation}
	\text{Recall} = \frac{TP}{TP + FN}
\end{equation}
Recall is also called sensitivity, is a fragment of the retrieved relevant instance \citet{smirani2022algorithm}.
\begin{equation}
	F1 = 2 \times \frac{\text{Precision} \times \text{Recall}}{\text{Precision} + \text{Recall}}
\end{equation}
F1-Score is a statistical measure that combines precision and recall with rate performance \citet{smirani2022algorithm}.

\begin{table}[H]
	\centering
	\renewcommand{\arraystretch}{1.3} % Adjust row height
	\caption{Classification Report with Precision, Recall, and F1-score}
	\begin{tabular}{|c|c|c|c|c|}
		\hline
		\textbf{Class} & \textbf{Precision} & \textbf{Recall} & \textbf{F1-score} & \textbf{Support} \\
		\hline
		\textbf{Dishonest} & 0.50 & 0.33 & 0.40 & 39 \\
		\textbf{Honest}  & 0.87 & 0.93 & 0.90 & 184 \\
		\hline
		\textbf{Accuracy}  & \multicolumn{4}{c|}{0.83} \\
		\hline
		\textbf{Macro Avg} & 0.68 & 0.63 & 0.65 & 223 \\
		\textbf{Weighted Avg} & 0.80 & 0.83 & 0.81 & 223 \\
		\hline
	\end{tabular}
	\label{tab:classification_report}
\end{table}

Table \ref{tab:classification_report} is the result of evaluating the classification model using Precision, Recall, F1-score, and Support metrics. This model is most likely to be used for anomaly detection.The specified limit or threshold, such as scores below 40.


\subsection{Log Monitoring}
In this research, a custom-developed log management dashboard was implemented to support the forensic readiness framework. Unlike commercial Security Information and Event Management (SIEM) solutions such as Splunk, IBM QRadar, or Elastic Stack (ELK), the custom dashboard was chosen due to its alignment with the goals of flexibility, lightweight deployment, educational accessibility, and forensic specificity.
\textbf{Key reasons for using a custom dashboard include:}

\begin{itemize}
	\item \textbf{Cost-efficiency:} SIEM platforms often require commercial licenses or high infrastructure costs, which are not feasible for academic environments or small institutions. A custom solution removes this barrier~\cite{barker2018openlogtools}.
	
	\item \textbf{Forensic Tailoring:} The custom dashboard was purpose-built to handle logs relevant to online examination systems, such as Moodle quiz attempts and user activity, which may not be natively supported or easily modeled in general-purpose SIEMs.
	
	\item \textbf{Transparency and Control:} Full access to the dashboard's source code and data pipelines allows greater transparency in log handling, which is critical for forensic validation and legal defensibility.
	
	\item \textbf{Lightweight and Focused:} SIEM tools often include broad and heavy telemetry modules not needed in this research. The custom dashboard uses a minimal stack (e.g., Flask, SQLite, or REST API) optimized for educational testing environments.
	
	\item \textbf{Ease of Integration with ML Models:} Integrating a machine learning-based anomaly detection model (e.g., Isolation Forest) directly into the custom dashboard is more straightforward than embedding it into a complex SIEM architecture.
	
	\item \textbf{Educational and Experimental Use:} In academic research, developing a tailored system allows hands-on experimentation with log structures, forensic workflows, and UI/UX designs—something not always possible with closed or semi-closed SIEMs.
\end{itemize}

Based on these reasons, a custom dashboard provides an ideal platform for validating forensic concepts in a resource-constrained environment while still ensuring critical functionality such as log analysis, visualization, classification, and anomaly alerting.

\begin{figure}[H] 
	\centering
	\includegraphics[width=16cm]{figure/log_attempt_step.png}
	\caption{Log Attempt Step}
	\label{fig:log-attempt-step}
\end{figure}
Figure~\ref{fig:log-attempt-step} illustrates the sequential log attempt steps recorded during an online examination session. Each log entry corresponds to a specific user action captured by the system and is categorized based on its execution state. These states represent the progression and outcome of each exam interaction, allowing forensic analysis to reconstruct user behavior.

The main states observed in the figure include \texttt{todo}, which signifies that the question was displayed to the participant but has not been answered; \texttt{complete}, indicating the participant submitted an answer; and two graded states: \texttt{gradedwrong} and \texttt{gradedright}, which show the automatic evaluation outcome of the submitted response. These log steps are timestamped and ordered, providing a temporal context for each transition.
\begin{figure}[H] 
	\centering
	\includegraphics[width=14cm]{figure/findings_20240809_215805.png}
	\caption{Finding user}
	\label{fig:findings-user}
\end{figure}

The figure \ref{fig:findings-user} to find case with user to indication cheating. On this page, there are users who are suspected of cheating. The data obtained will be stored in a database, including the timestamp and information on when the user committed the cheating. Therefore, the proctor will conduct another check after the exam.


\subsection{Notification}
In the development of the forensic log management framework, Telegram was selected as the primary alerting mechanism for several practical and technical reasons. While various alternatives exist for real-time notification systems such as email, Slack, Microsoft Teams, or SIEM integrated alerting the use of the Telegram Bot API offers a unique combination of simplicity, cost-effectiveness, and flexibility suitable for research and institutional deployments~\cite{wang2023telegramalert}.

\begin{figure}[H] 
	\centering
	\includegraphics[width=14cm]{figure/log-notification.jpeg}
	\caption{Notification}
	\label{fig:telegram-notification}
\end{figure}

Fig \ref{fig:telegram-notification} notification feature through Telegram bots is to facilitate better monitoring and response in an online examination system. Notification of the proactive analysis log results will be sent via telegram. 

However, the current implementation still requires manual intervention to trigger the notification to administrators or proctors. That is, although the log analysis system flags a suspicious case, an operator must manually confirm and forward the alert. This limitation reduces the level of automation in the incident response process. Future development should consider implementing a fully automated alerting mechanism, where notifications are sent immediately upon detection of a suspicious event.
\subsection{Log Preservation}
\begin{figure}[H] 
	\centering
	\includegraphics[width=14cm]{figure/log_preservation.png}
	\caption{Log Preservation Directory}
	\label{fig:log-preservation-1}
\end{figure}

In the figure \ref{fig:log-preservation-1}, it is a list that shows the date and timestamp information of activities performed by proactive log collection in the previous phase. Then, within it, there are three things obtained.

\begin{figure}[H] 
	\centering
	\includegraphics[width=14cm]{figure/log_preservation_detail.png}
	\caption{Detail Log Preservation}
	\label{fig:log-preservation-2}
\end{figure}

The display in the figure \ref{fig:log-preservation-2} shows the contents of log preservation. There are three logs that can be viewed: log attempt, log attempt step, and log general. Each log also has its hash value calculated to prevent log data changes or log tampering.

\begin{table}[H]
	\centering
	\caption{Example of standard\_log table structure}
	\label{tab:json_log_example}
	\begin{tabular}{|l|l|}
		\hline
		\textbf{Field} & \textbf{Value} \\ \hline
		action & reviewed \\ \hline
		component & mod\_quiz \\ \hline
		course\_name & EPrT HE Pre-Exam \\ \hline
		ip & 182.253.124.129 \\ \hline
		log\_id & 2514 \\ \hline
		quiz\_id & null \\ \hline
		quiz\_name & null \\ \hline
		target & attempt \\ \hline
		timecreated & 1736946654 \\ \hline
		user\_firstname & admin \\ \hline
		user\_id & 4 \\ \hline
		user\_lastname & admin \\ \hline
	\end{tabular}
\end{table}

\begin{itemize}
	\item action: User action (reviewed = viewing results)
	\item component: Related Moodle module (mod\_quiz = quiz)
	\item course\_name: Course name
	\item ip: User IP address (location tracking)
	\item log\_id: Unique log ID
	\item target: Action target (attempt = exam attempt)
	\item timecreated: Unix timestamp (seconds since 1/1/1970)
	\item user\_lastname: User identity (ID, name)
\end{itemize}

\begin{table}[htbp]
	\centering
	\caption{Quiz Attempt Log Data Example}
	\label{tab:quiz_attempt_log}
	\begin{tabular}{|l|l|}
		\hline
		\textbf{Field} & \textbf{Value} \\ \hline
		attempt\_id & 20 \\ \hline
		course\_name & EPrT HE Pre-Exam \\ \hline
		firstname & admin \\ \hline
		lastname & admin \\ \hline
		next\_step\_time & 1736946654 \\ \hline
		question\_attempt\_id & 104 \\ \hline
		quiz\_name & Grammar Pre-Exam \\ \hline
		score & 1.00000 \\ \hline
		step\_id & 296 \\ \hline
		step\_start\_time & 1736946642 \\ \hline
		step\_state & complete \\ \hline
		time\_spent\_on\_question & 12 (seconds) \\ \hline
		timefinish & 1736946654 \\ \hline
		timestart & 1736946639 \\ \hline
		uniqueid & 20 \\ \hline
		user\_id & 4 \\ \hline
	\end{tabular}
\end{table}

\begin{itemize}
	\item attempt\_id: 20 - Unique identifier for this quiz attempt
	\item course\_name: EPrT HE Pre-Exam - Name of the course containing the quiz
	\item firstname: admin - First name of the user who took the quiz
	\item lastname: admin - Last name of the user who took the quiz
	\item next\_step\_time: 1736946654 - Timestamp for when the next step should occur
	\item question\_attempt\_id: 104 - ID tracking this specific question attempt
	\item quiz\_name: Grammar Pre-Exam - Name of the quiz attempted
	\item score: 1.00000 - Points earned for this question (1.0)
	\item step\_id: 296 - Identifier for this step in the attempt
	\item step\_start\_time: 1736946642 - When this step began (Unix timestamp)
	\item step\_state: complete - Current status of this question step
	\item time\_spent\_on\_question: 12 - Seconds spent answering this question
	\item timefinish: 1736946654 - When attempt was completed
	\item timestart: 1736946639 - When attempt was started
	\item uniqueid: 20 - Another unique identifier for this attempt
	\item user\_id: 4 - Moodle's internal user identifier
\end{itemize}



\begin{table}[h]
	\centering
	\caption{Backup File Information}
	\label{tab:backup_info}
	\small
	\begin{tabular}{|p{1.3cm}|p{6.5cm}|}
		\hline
		\textbf{Field} & \textbf{Value} \\ \hline
		fullpath & \texttt{\scriptsize /home/.../2024-11-27-01:00:01/} \\ 
		& \texttt{\scriptsize mysql-dump-2024-11-27\_...\_log\_step.csv} \\ \hline
		md5 & \texttt{\footnotesize d41d8cd98f00b204e9800998ecf8427e} \\ \hline
		title & \texttt{\footnotesize mysql-dump-2024-11-27\_...\_log\_step.csv} \\ \hline
		url & \texttt{\scriptsize http://180.xxx.xxx.xxx:8443/.../} \\
		& \texttt{\scriptsize 2024-11-27-../mysql-dump-...\_log\_step.csv} \\ \hline
	\end{tabular}
\end{table}

The explanation of Table \ref{tab:backup_info} is outlined as follows:

\begin{itemize}
	\item \textbf{fullpath}: Complete server path to the backup CSV file containing MySQL log data
	\item \textbf{md5}: 32-character checksum for file verification (empty file indicator)
	\item \textbf{title}: Automated backup filename with timestamp
	\item \textbf{url}: Download link for retrieving the backup file
\end{itemize}

\subsection{Reporting}
\begin{figure}[H] 
	\centering
	\includegraphics[width=10cm]{figure/log_reporting_1.png}
	\caption{Reporting User}
	\label{fig:log_reporting_1}
\end{figure}

\begin{figure}[H] 
	\centering
	\includegraphics[width=10cm]{figure/log_reporting_2.png}
	\caption{Reporting User Download}
	\label{fig:log_reporting_2}
\end{figure}

The log reporting document serves as a consolidated summary of user activity during online examinations. It includes essential information such as the participant's name, timestamp of each recorded action, and a classification status indicating whether the behavior is considered suspicious. This structured report enables proctors or administrators to review potential anomalies efficiently and supports the decision-making process for further investigation or enforcement actions. The inclusion of timestamped events and flagged indicators enhances the traceability and forensic value of the evidence collected.


\section{Phase 5}
Phase 5 focuses on evaluating the implemented forensic log management framework through simulated forensic scenarios. This evaluation aims to determine whether the system is capable of supporting proactive forensic readiness—particularly in the context of online examination environments where incidents such as impersonation, unauthorized access, or rapid submission attempts may occur.
\subsection{Verification}
The results obtained from the experiment using the log management method revealed the phases from identification to reporting logs in online exams. The results can be seen in the table below.

\begin{table}[H]
	\centering
	\caption{Verifying testing framework adopted from NIST 800-92}
	\begin{tabular}{|p{3.2cm}|p{6cm}|p{4cm}|}
		\hline
		\textbf{Phase} & \textbf{Expected Result} & \textbf{Result} \\ \hline
		1. Log Identification & All log sources from Moodle and server are identified, including quiz attempts and activity logs & As Expected \\ \hline
		2. Log Proactive Collection & Automated scripts collect logs daily from exam sessions without disrupting system performance & As Expected \\ \hline
		3. Log Transmission & Log files transferred via \texttt{rsync} over SSH securely and consistently & As Expected \\ \hline
		4. Log Storage & Log data is stored in centralized, timestamped folders with access control and retention policy & As Expected \\ \hline
		5. Log Analyzer & Dashboard successfully displays log data with filtering, classification, and visualization features & As Expected \\ \hline
		6. Log Proactive Analysis & Anomaly detection using machine learning identifies suspicious patterns from log data & As Expected \\ \hline
		7. Send Notification & Telegram bot sends alert based on flagged anomalies from the dashboard to the administrator & As Expected \\ \hline
		8. Log Preservation & Logs are stored with integrity checks (MD5 hash) to ensure tamper-evidence & As Expected \\ \hline
		9. Log Reporting & PDF report is generated, presenting user activity and anomaly classification in structured format & As Expected \\ \hline
	\end{tabular}
	\label{tab:verifying_log_framework}
\end{table}


Table \ref{tab:verifying_log_framework} can be used to verify that the framework adopted from NIST 800-92 works as expected. The verification results of the framework show that each phase has been carried out according to its objectives and has achieved results that align with what was intended.



\section{Phase 6}
Phase 6 is the final stage in the development lifecycle, focusing on the validation of the proposed forensic log management framework by relevant domain experts. The purpose of this phase is to assess the framework’s technical soundness, practical applicability, and completeness when applied in real-world online examination environments.


\subsection{Validation}
To assess the effectiveness and practicality of the developed proactive forensic log management system, a validation process was conducted involving a panel of domain experts. The purpose of this validation was to evaluate the system's alignment with digital forensic principles, its technical reliability, and its suitability for use in online examination environments.

\textbf{Justification for Using Interviews:}  
The interview method was chosen because it enables deeper exploration of expert perspectives that may not be captured through quantitative surveys. Given the complexity and domain-specific nature of the system particularly in relation to log integrity, forensic readiness, and anomaly detection semi-structured interviews allow experts to elaborate on technical insights and provide contextual evaluations. This method is also appropriate for validating design decisions in early-stage frameworks where practical deployment feedback is critical.

\textbf{Main Questions Asked:}
\begin{itemize}
	\item Is the scope of log data collected sufficient for forensic investigation?
	\item Does the system maintain the integrity and traceability of log data effectively?
	\item How reliable is the anomaly detection approach based on machine learning?
	\item Are the notification and reporting features helpful for exam monitoring?
	\item Could this system be realistically deployed in a real online examination environment?
\end{itemize}

% \section{Log Grabbing or Log Identification}
% =========================================================
% Present the findings of the study in the order of the specific problem as stated in the statement of the Problem. Present the data in these forms: (a) tabular; (b) textual; and (c) graphical (optional). The ZOOM LENS approach may be used for purposes of clarity in the presentation of data, i.e. general to particular, macro to micro or vice versa. \textbf{(Note: Mean of data in here is data of the experiment result, not the data for input of the system)}
% Log identification dengan cara mengidentifikasi sumber yang menghasilkan log dengan cara melihat dari hasil peserta ujian. Hasil log attempt akan disimpan di dalam database oleh karena itu sumber lognya berada pada mysql database.




% =========================================================
% \section{Proactive Collection}



% \section{Log Transmission}


% \section{Log Storage}

% \section{Analysis of the Data}
% =========================================================


% \section{Log Monitoring}




% In thesis writing, the Chapter is simply a summary of what the researcher had done all throughout the whole research. 
% =========================================================

% \section{Notification}


% \section{Log Preservation}




% \section{Reporting}
% lorem ipsum

\section{Result of framework log management}


\begin{figure}[H] 
	\centering
	\includegraphics[width=14cm]{figure/log-management-nist-800-92-original.png}
	\caption{ Phase log management from NIST 800-92 2006 \citet{kentnist800922006guide}}
	\label{fig:nist-log-management}
\end{figure}

Fig \ref{fig:nist-log-management} original from publication nist 800-92 2006.
\begin{figure}[H] 
	\centering
	\includegraphics[width=14cm]{figure/proactive-collection.png}
	\caption{ Proactive Forensics \citet{proactiveandreactivedigitalforensics}}
	\label{fig:proactive-forensic}
\end{figure}


\begin{figure}[H]
	\centering
	\includegraphics[height=20cm]{figure/framework-adopted-nist-800-92.png}
	\caption{Proposed framework adapted from NIST 800-92}
	\label{fig:framework-proposed}
\end{figure}


Figure \ref{fig:framework-proposed} is the result of an adaptation based on NIST 800-92 regarding log management.The proposed framework adopted from NIST SP 800-92 (Guide to Computer Security Log Management) provides a structured approach to managing and analyzing logs.



% end
\section{Summary of Findings}
% =========================================================
% This describes the problem, research design, and the findings (answer to the questions raised). The recommended format is the paragraph form instead of the enumeration form. For each of the problems, present: (a) the salient findings; and (b) the results of the hypothesis tested.

\begin{table}[H]
	\centering
	\caption{Comparison of Proactive and Reactive Forensic Based on Log Management Aspects}
	\label{tab:forensic-log-comparison}
	\begin{tabular}{|p{4cm}|p{5.5cm}|p{5.5cm}|}
		\hline
		\textbf{Aspect} & \textbf{Proactive Forensic} & \textbf{Reactive Forensic} \\
		\hline
		\textbf{Log Generation} & Logging is configured in advance with consistent policies to capture all relevant events, even before incidents occur. & Logging may not be fully enabled until after an incident is detected or suspected. \\
		\hline
		\textbf{Log Transmission} & Logs are periodically transmitted to a centralized repository as part of ongoing readiness. & Logs may be manually collected from devices post-incident; transmission is reactive and possibly delayed. \\
		\hline
		\textbf{Log Storage} & Logs are stored securely with defined retention periods, using structured directories and integrity-preserving mechanisms. & Logs may be scattered or incomplete; storage begins or is prioritized after an incident is identified. \\
		\hline
		\textbf{Log Analysis} & Analysis is conducted before incidents occur, aiming to identify anomalies and potential threats early (proactive log analysis). & Analysis is difficult if logs are incomplete or missing; lack of evidence may hinder investigation. \\
		\hline
	\end{tabular}
\end{table}


Table \ref{tab:forensic-log-comparison} compares proactive and reactive forensic approaches based on key aspects of log management. In proactive forensics, log generation is pre-configured with consistent policies to ensure comprehensive event capture before incidents occur, whereas reactive forensics often lack complete logging until an incident is suspected. Proactive approaches also include scheduled log transmission to a centralized repository, secure storage with structured organization and integrity measures, and early-stage log analysis to detect anomalies. In contrast, reactive methods typically rely on delayed or manual log collection, ad-hoc storage, and limited analysis capabilities due to incomplete or missing logs, which can hinder effective investigations.


\begin{table}[H]
	\centering
	\caption{Log Management Processes}
	\label{tab:log-management-challanges-output-method}
	\begin{tabularx}{\textwidth}{p{0.8cm} p{2.8cm} p{3.2cm} X X}
		\toprule
		\textbf{No} & \textbf{Process} & \textbf{Method} & \textbf{Challenges} & \textbf{Output} \\
		\midrule
		1 & Log Identification & Identify all log sources & Legacy systems with non-standard formats (e.g., CSV, plaintext) & List of log sources \\
		2 & Log Proactive Collection & Automated daily backup scripts & Risk of database server overload during peak hours & Daily backups of quiz attempt logs \\
		3 & Log Transmission & File transfer using \texttt{rsync} protocol over SSH & Network latency, ensuring file consistency & Synchronized log files in log storage server \\
		4 & Log Storage & Centralized structured directory with access control & Scalability, retention policy enforcement, integrity preservation, and filtering irrelevant logs (e.g., unrelated Apache2 web server logs) & Organized, timestamped log archive \\
		5 & Log Analyzer & Custom dashboard development & Integration of multiple log formats into unified view & Log activity visualized via web dashboard \\
		6 & Log Proactive Analysis & Log anomaly detection using ML (Isolation Forest) & High computation and tuning threshold values & Preliminary anomaly classification \\
		7 & Send Notification & Telegram bot integration for alerts & Notification workflow still relies on manual confirmation & Cheating alert notification for administrators \\
		8 & Log Preservation & CSV export with MD5 hash checksum & Disk I/O load, ensuring file immutability & Verified and tamper-evident log files \\
		9 & Log Reporting & Document generation via dashboard export & Report standardization and formatting issues & PDF-based user activity reports \\
		\bottomrule
	\end{tabularx}
\end{table}


Table \ref{tab:log-management-challanges-output-method}, Log management initiates with Log Identification for comprehensive source mapping, frequently encountering interoperability issues with legacy system formats. Subsequent Log Proactive Collection utilizes automated daily backup mechanisms, presenting potential database server performance impacts. Analysis is conducted through a Custom Dashboard (Log Analyzer).



The log management phase is structured in the following sequence to support proactive forensic readiness:

\begin{enumerate}
	\item \textbf{Database Log Sources} \\
	The process begins with capturing high-relevance log data from the database, such as quiz attempts and user session activity, which serve as primary sources for digital forensic analysis.
	
	\item \textbf{Periodic Proactive Log Collection} \\
	Log data is periodically extracted from the database and other sources through automated mechanisms to ensure consistent and up-to-date monitoring of user activities.
	
	\item \textbf{Log Storage} \\
	Collected logs are then stored in a centralized, structured, and secure directory system, with mechanisms for timestamping, access control, and integrity verification.
\end{enumerate}