\chapter*{}
\vspace*{-1.0cm}
\begin{center}
\addcontentsline{toc}{chapter}{\protect\numberline{}ABSTRACT}
\normalfont\LARGE\textbf{ABSTRACT}
\end{center}
% This study presents a proactive forensic framework based on log management to improve forensic readiness and ensure the reliability of log evidence in online English Proficiency Test (EPT) environments. The framework incorporates automated log collection, organized storage, and near real-time analysis to address challenges related to evidence preservation and anomaly detection. Following the NIST 800-92 guidelines, the system ensures the secure handling and centralized management of logs, reducing the risk of data loss. By integrating proactive monitoring and reporting features, such as automated alerts and anomaly detection dashboards, the system facilitates prompt identification and response to suspicious activities. The results highlight the framework's effectiveness in simplifying the forensic investigation process, enhancing log utility, and maintaining the integrity of online examinations. This approach contributes to the advancement of forensic readiness in educational institutions, providing a scalable and efficient solution for maintaining the credibility of digital evidence.\\\\
% \textbf{Keywords:} Proactive Forensics, Log Management, Online Exams, Forensic Readiness, NIST 800-92, English Proficiency Test (EPT)