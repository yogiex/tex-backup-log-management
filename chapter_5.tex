\chapter{Conclusion and Recommendations}

\section{Conclusion}

This study demonstrates that proactive forensics is a viable approach for enhancing the integrity and auditability of online examination systems. By enabling the collection of log data prior to the occurrence of suspicious activities, proactive forensics ensures the availability and reliability of digital evidence for further analysis. The proactive collection process is central to this capability, as it facilitates continuous monitoring and automated acquisition of log data, including user interactions, quiz attempts, and system-generated records from the Moodle platform.

The implementation of a framework based on NIST Special Publication 800-92 provides structured guidance in managing logs systematically. Integrating this standard with proactive forensic techniques improves the readiness and responsiveness of digital forensic activities, especially in academic environments. This integration supports secure log acquisition, centralized storage, anomaly detection, and evidence preservation.

Furthermore, the application of machine learning, particularly anomaly detection using Isolation Forest, enhances the ability to identify potentially fraudulent behavior that might be overlooked through manual inspection. Overall, the proposed system contributes to the advancement of forensic readiness by combining structured log management with intelligent analysis mechanisms.
\begin{table}[H]
	\centering
	\caption{Examples of Logging Configuration Settings (Adapted from NIST SP 800-92)}
	\begin{tabular}{|p{5.5cm}|p{3cm}|p{3cm}|p{3cm}|}
		\hline
		\textbf{Category} & \textbf{Low Impact Systems} & \textbf{Moderate Impact Systems} & \textbf{High Impact Systems} \\
		\hline
		How long to retain log data & 1 to 2 weeks & 1 to 3 months & 3 to 12 months \\
		\hline
		How often to rotate logs & Optional (if performed, at least weekly or every 25 MB) & Every 6 to 24 hours, or every 2 to 5 MB & Every 15 to 60 minutes, or every 0.5 to 1.0 MB \\
		\hline
		Frequency of transferring log data to the log management infrastructure & Every 3 to 24 hours & Every 15 to 60 minutes & At least every 5 minutes \\
		\hline
		How often log data needs to be analyzed & Every 1 to 7 days & Every 12 to 24 hours & At least 6 times a day \\
		\hline
		Log file integrity checking & Optional & Yes & Yes \\
		\hline
		Encryption of rotated logs & Optional & Optional & Yes \\
		\hline
		Whether log data transfers need to be encrypted or performed on a separate logging network & Optional & Yes, if feasible & Yes \\
		\hline
	\end{tabular}
	\label{tab:logging_config_nist}
\end{table}

Table~\ref{tab:logging_config_nist} shows the recommended log management configurations according to NIST SP 800-92. These benchmarks serve as reference criteria for evaluating the proposed framework. Based on the analysis, our implementation achieves log rotation, integrity checking, and transfer intervals that align with the requirements for moderate to high-impact systems.

\section{Recommendations}

Based on the findings and conclusions of this study, several recommendations are proposed to support further development and application of proactive forensic systems:

\begin{itemize}
    \item Future work should explore the integration of alternative log management frameworks or technologies to improve adaptability and performance in various academic settings.
    \item Comparative studies with other proactive forensic techniques should be conducted to evaluate effectiveness and scalability in broader deployment scenarios.
    \item Since this research was conducted as a controlled prototype simulation, it is recommended that subsequent implementations be tested in live academic environments to validate system robustness and practical utility.
    \item Additional work could also focus on enhancing the anomaly detection mechanism by experimenting with different machine learning models or incorporating more granular behavioral metrics.
\end{itemize}

These recommendations are intended to support future research and practical implementation efforts toward achieving a robust, scalable, and forensic-ready online examination system.
