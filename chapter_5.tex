\chapter{Conclusion and Recommendations}

\section{Conclusion}

This study demonstrates that proactive forensics is a viable approach for enhancing the integrity and auditability of online examination systems. By enabling the collection of log data prior to the occurrence of suspicious activities, proactive forensics ensures the availability and reliability of digital evidence for further analysis. The proactive collection process is central to this capability, as it facilitates continuous monitoring and automated acquisition of log data, including user interactions, quiz attempts, and system-generated records from the Moodle platform.

The implementation of a framework based on NIST Special Publication 800-92 provides structured guidance in managing logs systematically. Integrating this standard with proactive forensic techniques improves the readiness and responsiveness of digital forensic activities, especially in academic environments. This integration supports secure log acquisition, centralized storage, anomaly detection, and evidence preservation.

Furthermore, the application of machine learning, particularly anomaly detection using Isolation Forest, enhances the ability to identify potentially fraudulent behavior that might be overlooked through manual inspection. Overall, the proposed system contributes to the advancement of forensic readiness by combining structured log management with intelligent analysis mechanisms.

\section{Recommendations}

Based on the findings and conclusions of this study, several recommendations are proposed to support further development and application of proactive forensic systems:

\begin{itemize}
    \item Future work should explore the integration of alternative log management frameworks or technologies to improve adaptability and performance in various academic settings.
    \item Comparative studies with other proactive forensic techniques should be conducted to evaluate effectiveness and scalability in broader deployment scenarios.
    \item Since this research was conducted as a controlled prototype simulation, it is recommended that subsequent implementations be tested in live academic environments to validate system robustness and practical utility.
    \item Additional work could also focus on enhancing the anomaly detection mechanism by experimenting with different machine learning models or incorporating more granular behavioral metrics.
\end{itemize}

These recommendations are intended to support future research and practical implementation efforts toward achieving a robust, scalable, and forensic-ready online examination system.
